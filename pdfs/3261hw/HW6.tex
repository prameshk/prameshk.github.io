\documentclass[letterpaper,11pt,twoside]{article}

\usepackage[left=1in, right=1in, bottom=1.25in, top=1.5in]{geometry}
\usepackage[utf8]{inputenc}
\usepackage{setspace}
%\usepackage{hyperref}
\usepackage{fancyhdr}
\usepackage{amsmath}
\usepackage{amsfonts}
\usepackage{amssymb}
\usepackage{amsthm}
\allowdisplaybreaks
\usepackage[T1]{fontenc}
\usepackage{xcolor}
\usepackage[mathscr]{euscript}
\usepackage{latexsym,bbm,xspace,graphicx,float,mathtools,mathdots,xspace}
\usepackage{enumitem}
\usepackage[ruled,vlined]{algorithm2e}
\usepackage{bm}
\usepackage[backref, colorlinks,citecolor=blue,linkcolor=magenta,bookmarks=true]{hyperref}
\usepackage[nameinlink]{cleveref}

% tikz
\usepackage{subfig}
\usepackage{graphicx}
\usepackage{tikz}

\usepackage{tablefootnote}

\fancypagestyle{plain}{%
\fancyhf{} % clear all header and footer fields
\fancyfoot[C]{\textbf{\thepage}} % except the center
\renewcommand{\headrulewidth}{0pt}
\renewcommand{\footrulewidth}{0pt}}

\theoremstyle{plain}
\newtheorem{theorem}{Theorem}
\newtheorem{assumption}{Assumption}
\newtheorem{corollary}{Corollary}
\newtheorem{lemma}{Lemma}
\newtheorem{conjecture}{Conjecture}
\newtheorem{proposition}{Proposition}
\newtheorem{observation}{Observation}
\newtheorem{claim}{Claim}
\newtheorem{property}{Property}
\newtheorem{op}{Open Problem}
\newtheorem{problem}{Problem}
\newtheorem{question}{Question}

\theoremstyle{definition}
\newtheorem{definition}{Definition}
\newtheorem{example}{Example}
\newtheorem{sketch}{Sketch}
\newtheorem{idea}{Idea}

\theoremstyle{remark}
\newtheorem{remark}{Remark}

\newtheoremstyle{restate}{}{}{\itshape}{}{\bfseries}{~(restated).}{.5em}{\thmnote{#3}}
\theoremstyle{restate}
\newtheorem*{restate}{}



\crefname{theorem}{Theorem}{Theorems}
\crefname{assumption}{Assumption}{Assumptions}
\crefname{corollary}{Corollary}{Corollaries}
\crefname{lemma}{Lemma}{Lemmas}
\crefname{conjecture}{Conjecture}{Conjectures}
\crefname{proposition}{Proposition}{Propositions}
\crefname{observation}{Observation}{Observations}
\crefname{claim}{Claim}{Claims}
\crefname{property}{Property}{Properties}
\crefname{op}{Open Problem}{Open Problems}
\crefname{problem}{Problem}{Problems}
\crefname{question}{Question}{Questions}

% \crefname{fact}{Fact}{Facts}

\crefname{definition}{Definition}{Definitions}
\crefname{example}{Example}{Examples}
\crefname{sketch}{Sketch}{Sketches}
\crefname{idea}{Idea}{Ideas}

% \crefname{condition}{Condition}{Conditions}

\crefname{remark}{Remark}{Remarks}



\crefname{equation}{Equation}{Equations}
\crefname{figure}{Figure}{Figures}
\crefname{table}{Table}{Tables}



%%%%%%%%%%%%%%%%%%%%%%%%%%
% GENERAL-PURPOSE MACROS %
%%%%%%%%%%%%%%%%%%%%%%%%%%
\newcommand{\uth}{\bigskip \bigskip {\huge {\red{UP TO HERE}} \bigskip \bigskip}}
\newcommand{\ignore}[1]{}
\newcommand{\eps}{\varepsilon}
\newcommand{\simple}{\mathrm{simple}}
\newcommand{\E}{\operatorname{{\bf E}}}
\newcommand{\Ex}{\mathop{{\bf E}\/}}
\renewcommand{\Pr}{\operatorname{{\bf Pr}}}
\newcommand{\Prx}{\mathop{{\bf Pr}\/}}
\newcommand{\Var}{\operatorname{{\bf Var}}}
\newcommand{\Varx}{\mathop{{\bf Var}\/}}
\newcommand{\tO}{\tilde{O}}
\newcommand{\sgn}{\mathrm{sgn}}
\newcommand{\X}{\mathcal{X}}
\newcommand{\Y}{\mathcal{Y}}
\newcommand{\rZ}{\mathcal{Z}}

\DeclareMathOperator\erf{erf}

\newcommand{\polylog}{\mathrm{polylog}}
\newcommand{\poly}{\mathrm{poly}}
\newcommand{\bcalZ}{\bm{\mathcal{Z}}}
\newcommand{\bx}{\bm{x}}
\newcommand{\by}{\bm{y}}
\newcommand{\bxi}{\bm{\xi}}

%%%%%%%%%%%%%%%%%%
% NUMBER SYSTEMS %
%%%%%%%%%%%%%%%%%%
\newcommand{\R}{\mathbb R}
\newcommand{\RR}{\R_{\geq 0}}
\newcommand{\C}{\mathbb C}
\newcommand{\N}{\mathbb N}
\newcommand{\NN}{\N_{\geq 1}}
\newcommand{\Z}{\mathbb Z}

\renewcommand{\i}{\mathbf{i}}   % for complex numbers
\renewcommand{\d}{\mathrm{d}}   % for integrals
\newcommand{\lhs}{\mathrm{LHS}} % for inequalities
\newcommand{\rhs}{\mathrm{RHS}} % for inequalities
\newcommand{\supp}{\mathrm{supp}}
\renewcommand{\hat}[1]{\widehat{#1}}
\renewcommand{\bar}[1]{\overline{#1}}
\newcommand{\sig}{\mathrm{sig}}

\newcommand{\comment}[1]{}

% Define colors
\def\colorful{1}
\ifnum\colorful=1
\newcommand{\violet}[1]{{\color{violet}{#1}}}
\newcommand{\orange}[1]{{\color{orange}{#1}}}
\newcommand{\blue}[1]{{{\color{blue}#1}}}
\newcommand{\red}[1]{{\color{red} {#1}}}
\newcommand{\green}[1]{{\color{green} {#1}}}
\newcommand{\pink}[1]{{\color{pink}{#1}}}
\newcommand{\gray}[1]{{\color{gray}{#1}}}
\fi
\ifnum\colorful=0
\newcommand{\violet}[1]{{{#1}}}
\newcommand{\orange}[1]{{{#1}}}
\newcommand{\blue}[1]{{{#1}}}
\newcommand{\red}[1]{{{#1}}}
\newcommand{\green}[1]{{{#1}}}
\newcommand{\gray}[1]{{{#1}}}
\fi

\title{Homework 6}
% \author{Tim Randolph}
\date{COMS W3261, Summer B 2021}

\begin{document}

\maketitle

This homework is due \textbf{Monday, 7/19/2021, at 11:59PM EST}. Submit to GradeScope (course code: X3JEX4).

Grading policy reminder: \LaTeX~is preferred, but neatly typed or handwritten solutions are acceptable. Your TAs may dock points for indecipherable writing. Proofs should be complete; that is, include enough information that a reader can clearly tell that the argument is rigorous.

The tool \url{http://madebyevan.com/fsm/} may be useful for drawing finite state machines.

If a question is ambiguous, please state your assumptions. This way, we can give you credit for correct work. (Even better, post on Ed so that we can resolve the ambiguity.)

\clearpage
\section{Problem 1 (12 points)}

\begin{enumerate}
    \item (4 points). Consider a variant Turing Machine with a transition function
    \[
        \delta : Q \times \Gamma \rightarrow Q \times \Gamma \times \{3L, 2R\},
    \]
    where the instructions $3L$ and $2R$ move the tape head three spaces to the left (halting if we reach the leftmost tape square) and two spaces to the right, respectively. Show that this variant TM is equivalent to an ordinary TM by explaining how to simulate it with an ordinary TM and vice versa. (For this question, an implementation-level description of head movement is sufficient: you do not need to formally specify the simulating TM.)

    \item (8 points). Consider a variant Turing Machine that reads and writes on an infinite 2-dimensional tape. Each tape square is indexed by a pair of non-negative integers (i.e., we can think of the tape as beginning at an origin square (0,0) and extending into the first quadrant of the coordinate grid.) The TM begins execution with the tape head at $(0,0)$ and the  generic input string $w = w_0w_1w_2\dots w_n$ written on the first row of the tape (i.e., squares $(0,0), (1,0), \dots, (n,0)$). This variant TM has the transition function
    \[
        \delta : Q \times \Gamma \rightarrow Q \times \Gamma \times \{L, R, U, D\},
    \]
    where the symbols $U$ and $D$ move the tape head up and down, respectively (and the tape head cannot go below 0 in either coordinate).
    
    Show that this variant TM is equivalent to an ordinary TM by explaining how to simulate it with an ordinary TM and vice versa.  (For this question, a description of memory management is sufficient: you do not need to formally specify the simulating TM or describe the precise details of head movement.)
\end{enumerate}

\clearpage
\section{Problem 2 (8 points)}
\begin{enumerate}
    \item (3 points). Prove that the language
    \[
        ALL_{DFA} := \{ \langle A \rangle \; | \; A \text{ is a DFA and } L(A) = \Sigma^*\}
    \]
    is decidable by giving a high-level description of a Turing Machine that decides $ALL_{DFA}$.
    
    \item (5 points). Prove that the language
    \[
        Y_1 := \{ \langle A \rangle \; | \; A \text{ is a DFA and } L(A) \text{ contains all strings over } \{0,1\} \text{ with at least one 1.} \}
    \]
    is decidable. (Hint: you can refer to the fact that $ALL_{DFA}$ is decidable.)
    
\end{enumerate}

\clearpage
\section{Problem 3 (5 points)}
    \begin{enumerate}
    \item (5 points.) Show that the language 
    \[
        ALL_{TM} := \{ \langle M \rangle \; | \; M \text{ is a Turing Machine and } L(M) = \Sigma^*\}
    \]
    is undecidable by reducing $A_{TM}$ to $ALL_{TM}$. (That is, show that if you could decide this problem, you could decide $A_{TM}$. Conclude that because $A_{TM}$ is undecidable, $ALL_{TM}$ must be undecidable.)
    
    \end{enumerate}
    
    
\clearpage
\section{Problem 4 (Extra credit, 3 points)}
\begin{enumerate}
    \item (3 points). Which two previous homework problems did you find most challenging? Download the solutions from the course website, look them over, and bring any questions to TA/office hours. (For Problem Sets 5 and 6, go directly to office hours.) Good luck on the final!
    
\end{enumerate}
\end{document}