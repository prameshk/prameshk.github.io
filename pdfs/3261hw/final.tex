\documentclass[letterpaper,11pt,twoside]{article}

\usepackage[left=1in, right=1in, bottom=1.25in, top=1.5in]{geometry}
\usepackage[utf8]{inputenc}
\usepackage{setspace}
%\usepackage{hyperref}
\usepackage{fancyhdr}
\usepackage{amsmath}
\usepackage{amsfonts}
\usepackage{amssymb}
\usepackage{amsthm}
\allowdisplaybreaks
\usepackage[T1]{fontenc}
\usepackage{xcolor}
\usepackage[mathscr]{euscript}
\usepackage{latexsym,bbm,xspace,graphicx,float,mathtools,mathdots,xspace}
\usepackage{enumitem}
\usepackage[ruled,vlined]{algorithm2e}
\usepackage{bm}
\usepackage[backref, colorlinks,citecolor=blue,linkcolor=magenta,bookmarks=true]{hyperref}
\usepackage[nameinlink]{cleveref}

% tikz
\usepackage{subfig}
\usepackage{graphicx}
\usepackage{tikz}

\usepackage{tablefootnote}

\fancypagestyle{plain}{%
\fancyhf{} % clear all header and footer fields
\fancyfoot[C]{\textbf{\thepage}} % except the center
\renewcommand{\headrulewidth}{0pt}
\renewcommand{\footrulewidth}{0pt}}

\theoremstyle{plain}
\newtheorem{theorem}{Theorem}
\newtheorem{assumption}{Assumption}
\newtheorem{corollary}{Corollary}
\newtheorem{lemma}{Lemma}
\newtheorem{conjecture}{Conjecture}
\newtheorem{proposition}{Proposition}
\newtheorem{observation}{Observation}
\newtheorem{claim}{Claim}
\newtheorem{property}{Property}
\newtheorem{op}{Open Problem}
\newtheorem{problem}{Problem}
\newtheorem{question}{Question}

\theoremstyle{definition}
\newtheorem{definition}{Definition}
\newtheorem{example}{Example}
\newtheorem{sketch}{Sketch}
\newtheorem{idea}{Idea}

\theoremstyle{remark}
\newtheorem{remark}{Remark}

\newtheoremstyle{restate}{}{}{\itshape}{}{\bfseries}{~(restated).}{.5em}{\thmnote{#3}}
\theoremstyle{restate}
\newtheorem*{restate}{}



\crefname{theorem}{Theorem}{Theorems}
\crefname{assumption}{Assumption}{Assumptions}
\crefname{corollary}{Corollary}{Corollaries}
\crefname{lemma}{Lemma}{Lemmas}
\crefname{conjecture}{Conjecture}{Conjectures}
\crefname{proposition}{Proposition}{Propositions}
\crefname{observation}{Observation}{Observations}
\crefname{claim}{Claim}{Claims}
\crefname{property}{Property}{Properties}
\crefname{op}{Open Problem}{Open Problems}
\crefname{problem}{Problem}{Problems}
\crefname{question}{Question}{Questions}

% \crefname{fact}{Fact}{Facts}

\crefname{definition}{Definition}{Definitions}
\crefname{example}{Example}{Examples}
\crefname{sketch}{Sketch}{Sketches}
\crefname{idea}{Idea}{Ideas}

% \crefname{condition}{Condition}{Conditions}

\crefname{remark}{Remark}{Remarks}



\crefname{equation}{Equation}{Equations}
\crefname{figure}{Figure}{Figures}
\crefname{table}{Table}{Tables}



%%%%%%%%%%%%%%%%%%%%%%%%%%
% GENERAL-PURPOSE MACROS %
%%%%%%%%%%%%%%%%%%%%%%%%%%
\newcommand{\uth}{\bigskip \bigskip {\huge {\red{UP TO HERE}} \bigskip \bigskip}}
\newcommand{\ignore}[1]{}
\newcommand{\eps}{\varepsilon}
\newcommand{\simple}{\mathrm{simple}}
\newcommand{\E}{\operatorname{{\bf E}}}
\newcommand{\Ex}{\mathop{{\bf E}\/}}
\renewcommand{\Pr}{\operatorname{{\bf Pr}}}
\newcommand{\Prx}{\mathop{{\bf Pr}\/}}
\newcommand{\Var}{\operatorname{{\bf Var}}}
\newcommand{\Varx}{\mathop{{\bf Var}\/}}
\newcommand{\tO}{\tilde{O}}
\newcommand{\sgn}{\mathrm{sgn}}
\newcommand{\X}{\mathcal{X}}
\newcommand{\Y}{\mathcal{Y}}
\newcommand{\rZ}{\mathcal{Z}}

\DeclareMathOperator\erf{erf}

\newcommand{\polylog}{\mathrm{polylog}}
\newcommand{\poly}{\mathrm{poly}}
\newcommand{\bcalZ}{\bm{\mathcal{Z}}}
\newcommand{\bx}{\bm{x}}
\newcommand{\by}{\bm{y}}
\newcommand{\bxi}{\bm{\xi}}

%%%%%%%%%%%%%%%%%%
% NUMBER SYSTEMS %
%%%%%%%%%%%%%%%%%%
\newcommand{\R}{\mathbb R}
\newcommand{\RR}{\R_{\geq 0}}
\newcommand{\C}{\mathbb C}
\newcommand{\N}{\mathbb N}
\newcommand{\NN}{\N_{\geq 1}}
\newcommand{\Z}{\mathbb Z}

\renewcommand{\i}{\mathbf{i}}   % for complex numbers
\renewcommand{\d}{\mathrm{d}}   % for integrals
\newcommand{\lhs}{\mathrm{LHS}} % for inequalities
\newcommand{\rhs}{\mathrm{RHS}} % for inequalities
\newcommand{\supp}{\mathrm{supp}}
\renewcommand{\hat}[1]{\widehat{#1}}
\renewcommand{\bar}[1]{\overline{#1}}
\newcommand{\sig}{\mathrm{sig}}

\newcommand{\comment}[1]{}

% Define colors
\def\colorful{1}
\ifnum\colorful=1
\newcommand{\violet}[1]{{\color{violet}{#1}}}
\newcommand{\orange}[1]{{\color{orange}{#1}}}
\newcommand{\blue}[1]{{{\color{blue}#1}}}
\newcommand{\red}[1]{{\color{red} {#1}}}
\newcommand{\green}[1]{{\color{green} {#1}}}
\newcommand{\pink}[1]{{\color{pink}{#1}}}
\newcommand{\gray}[1]{{\color{gray}{#1}}}
\fi
\ifnum\colorful=0
\newcommand{\violet}[1]{{{#1}}}
\newcommand{\orange}[1]{{{#1}}}
\newcommand{\blue}[1]{{{#1}}}
\newcommand{\red}[1]{{{#1}}}
\newcommand{\green}[1]{{{#1}}}
\newcommand{\gray}[1]{{{#1}}}
\fi

\title{Computer Science Theory Final Exam}
% \author{Tim Randolph}
\date{COMS W3261, Summer B 2021}

\begin{document}

\maketitle

This exam will be available from 12:01 AM EST on Tuesday, 8/10/2021 until Wednesday, 8/11/2021, at 11:59PM EST. You have \textbf{12 hours} to complete this exam from the time you download it; you can use any 12 hours in the 48-hour window to work on the exam. The exam is due \textbf{Wednesday, 8/11/2021, at 11:59PM EST with NO EXTENSIONS}. Submit to GradeScope (course code: X3JEX4). 

Grading policy reminder: \LaTeX~is preferred, but neatly typed or handwritten solutions are acceptable. Your TAs may dock points for indecipherable writing. Proofs should be complete; that is, include enough information that a reader can clearly tell that the argument is rigorous.

If a question is ambiguous, please state your assumptions. This way, we can give you credit for correct work. (Even better, post on Ed so that we can resolve the ambiguity.)

You can consult the following resources during the exam:
\begin{enumerate}
    \item Your notes and class notes posted on the course webpage.
    \item Lecture recordings on YouTube and in-class recordings in the Zoom recordings folder on CourseWorks.
    \item Past homework problems, your homework answers, and solutions posted on the course webpage.
    \item The textbook.
    \item Past questions posted on Ed. (You may post on Ed during the exam, but please do not answer other student's questions or provide any details that might help others solve a problem. Answers from instructors will be limited to basic typo correction and clarifications only.)
    \item Resources for \LaTeX~formatting that are unrelated to the course material.
\end{enumerate}
Any resources not listed here, including all reference sources beyond the textbook, are not permitted.

\clearpage
\section{Problem 1 (10 points)}
    For this problem, consider the NFA state diagram pictured below.
    \begin{center}
    \begin{tikzpicture}[scale=0.2]
    \tikzstyle{every node}+=[inner sep=0pt]
    \draw [black] (49.1,-35.7) circle (3);
    \draw (49.1,-35.7) node {$q_6$};
    \draw [black] (38.9,-35.7) circle (3);
    \draw (38.9,-35.7) node {$q_5$};
    \draw [black] (38.9,-35.7) circle (2.4);
    \draw [black] (29.3,-26.8) circle (3);
    \draw (29.3,-26.8) node {$q_1$};
    \draw [black] (39.2,-16.9) circle (3);
    \draw (39.2,-16.9) node {$q_2$};
    \draw [black] (49.1,-17.1) circle (3);
    \draw (49.1,-17.1) node {$q_3$};
    \draw [black] (59.1,-17.1) circle (3);
    \draw (59.1,-17.1) node {$q_4$};
    \draw [black] (59.1,-17.1) circle (2.4);
    \draw [black] (18.9,-26.8) circle (3);
    \draw (18.9,-26.8) node {$q_0$};
    \draw [black] (18.9,-26.8) circle (2.4);
    \draw [black] (41.308,-33.952) arc (113.33544:66.66456:6.796);
    \fill [black] (46.69,-33.95) -- (46.16,-33.18) -- (45.76,-34.09);
    \draw (44,-32.9) node [above] {$b$};
    \draw [black] (47.061,-37.855) arc (-58.20675:-121.79325:5.81);
    \fill [black] (40.94,-37.86) -- (41.36,-38.7) -- (41.88,-37.85);
    \draw (44,-39.23) node [below] {$a$};
    \draw [black] (42.2,-16.96) -- (46.1,-17.04);
    \fill [black] (46.1,-17.04) -- (45.31,-16.52) -- (45.29,-17.52);
    \draw (44.14,-17.52) node [below] {$0$};
    \draw [black] (47.777,-14.42) arc (234:-54:2.25);
    \draw (49.1,-9.85) node [above] {$0$};
    \fill [black] (50.42,-14.42) -- (51.3,-14.07) -- (50.49,-13.48);
    \draw [black] (52.1,-17.1) -- (56.1,-17.1);
    \fill [black] (56.1,-17.1) -- (55.3,-16.6) -- (55.3,-17.6);
    \draw (54.1,-17.6) node [below] {$1$};
    \draw [black] (56.932,-19.172) arc (-48.89099:-95.04853:33.078);
    \fill [black] (32.27,-27.2) -- (33.03,-27.77) -- (33.11,-26.77);
    \draw (46.23,-26.24) node [below] {$\epsilon$};
    \draw [black] (31.8,-28.456) arc (53.74354:40.59029:31.311);
    \fill [black] (37.06,-33.33) -- (36.92,-32.4) -- (36.16,-33.05);
    \draw (35.51,-30.26) node [above] {$\epsilon$};
    \draw [black] (31.42,-24.68) -- (37.08,-19.02);
    \fill [black] (37.08,-19.02) -- (36.16,-19.23) -- (36.87,-19.94);
    \draw (34.77,-23.33) node [right] {$\epsilon$};
    \draw [black] (21.9,-26.8) -- (26.3,-26.8);
    \fill [black] (26.3,-26.8) -- (25.5,-26.3) -- (25.5,-27.3);
    \draw (24.1,-27.3) node [below] {$\epsilon$};
    \draw [black] (13.1,-26.8) -- (15.9,-26.8);
    \fill [black] (15.9,-26.8) -- (15.1,-26.3) -- (15.1,-27.3);
    \draw [black] (35.938,-36.026) arc (-95.15868:-170.50749:7.517);
    \fill [black] (29.2,-29.78) -- (28.84,-30.65) -- (29.82,-30.49);
    \draw (30.57,-34.54) node [below] {$\epsilon$};
    \end{tikzpicture}
    \end{center}

    
    \begin{enumerate}
        \item (3 points.) Does this NFA accept the empty string $\varepsilon$? What about the string 00101? What about the string ba0011?
        
        \item (2 points.) In class, we showed that the class of regular languages was closed under Kleene star ($^*$) by demonstrating that, given an NFA $N$ that recognizes any language $A$, we can convert $N$ into a new NFA $N'$ that recognizes $A^*$. The NFA pictured above was created by this conversion process. Which transition(s) or state(s) may have been added during conversion?
        
        \item (5 points.) Write a regular expression that evaluates to the same language that the NFA recognizes. Explain in a few sentences why the expression and the NFA represent the same language. (Hint: begin by reverse-engineering the conversion process for $^*$.)
    \end{enumerate}
    
\clearpage
\section{Problem 2 (14 points)}
    For this problem, consider the language
    \[
        L_1 = \{0^i1^{i+j}0^j \; | \; i, j \geq 0 \}.
    \]
    \begin{enumerate}
        \item (7 points.) Use the pumping lemma for regular languages to prove that $L$ is nonregular.
        
        \item (7 points.) Prove that $L$ is context-free by creating a context-free grammar that generates $L$. Explain in a few sentences why your CFG generates exactly the strings in $L$.
    \end{enumerate}

\clearpage
\section{Problem 3 (8 points)}
    Recall that the XOR operator $\oplus$ evaluates to $1$ if exactly one of its inputs is 1: thus $0 \oplus 0 = 0$, $0 \oplus 1 = 1$, $1 \oplus 0 = 1$, $1 \oplus 1 = 0$. Given two binary strings  $w, v \in \{0,1\}^*$ that have the same length, the binary XOR of $w$ and $v$, written $w \oplus v$, is the string $x$ such that $w_i \oplus v_i = x_i$ for all $i$. For example,
    \[
        01001 \oplus 11000 = 10001.
    \]
    Consider the language 
    \[
        L_2 = \{ a \# b \# c \; | \; a, b, c \in \{0,1\}^* \text{ and } a \oplus b = c \}.
    \]
    \begin{enumerate}
        \item (8 points.) Prove that $L_2$ is decidable by writing an implementation-level description of a Turing machine that decides $L_2$. (Don't forget to check that your input is correctly formatted.) 
    \end{enumerate}

\clearpage
\section{Problem 4 (8 points)}
    \begin{enumerate}
        \item (4 points.) Prove that the language
        \[
            L_3 = \{ a \; | \; a \text{ is a prime number that is also the sum of two square numbers} \}
        \]
        is decidable by writing a high-level description of a Turing machine that decides $L_3$.
        
        \item (4 points.) Prove that the language
        \[
            L_4 = \{ \langle A \rangle \; | \; A \text{ is a Turing machine that halts on \emph{any} input string } w \in \Sigma^*\}
        \]
        is recognizable by writing a high-level description of a Turing machine that recognizes $L_4$.
    \end{enumerate}
    
    
\clearpage
\section{Problem 5 (10 points)}
    \begin{enumerate}
        \item Prove that the language
        \[
            L_5 = \{ \langle A, B \rangle \; | \; A \text{ and } B \text{ are Turing machines and } L(A) \subseteq L(B) \}
        \]
        is undecidable by reducing another undecidable language to $L_5$. You may not use Rice's theorem. (Hint: it suffices to show that \emph{if} you could build a decider for $L_5$, you could build a decider for a language that is proved undecidable in lecture, in homework or in the textbook.)
    \end{enumerate}
    
\clearpage
\section{Problem 6 (3 extra credit points)}
    \begin{enumerate}
        \item (3 points). Consider the following language:
        \[
            \overline{SELF_{ENUM}} = \{\langle E \rangle \; | \; E \text{ is an enumerator that never prints out its own description } \langle E \rangle\}.
        \]
        Recall that a language is Turing-recognizable if and only if some enumerator enumerates it. Show that $\overline{SELF_{ENUM}}$ is not Turing-recognizable by finding a paradox.
    \end{enumerate}
\end{document}


