\documentclass[letterpaper,11pt,twoside]{article}

\usepackage[left=1in, right=1in, bottom=1.25in, top=1.5in]{geometry}
\usepackage[utf8]{inputenc}
\usepackage{setspace}
%\usepackage{hyperref}
\usepackage{fancyhdr}
\usepackage{amsmath}
\usepackage{amsfonts}
\usepackage{amssymb}
\usepackage{amsthm}
\allowdisplaybreaks
\usepackage[T1]{fontenc}
\usepackage{xcolor}
\usepackage[mathscr]{euscript}
\usepackage{latexsym,bbm,xspace,graphicx,float,mathtools,mathdots,xspace}
\usepackage{enumitem}
\usepackage[ruled,vlined]{algorithm2e}
\usepackage{bm}
\usepackage[backref, colorlinks,citecolor=blue,linkcolor=magenta,bookmarks=true]{hyperref}
\usepackage[nameinlink]{cleveref}

% tikz
\usepackage{subfig}
\usepackage{graphicx}
\usepackage{tikz}

\usepackage{tablefootnote}

\fancypagestyle{plain}{%
\fancyhf{} % clear all header and footer fields
\fancyfoot[C]{\textbf{\thepage}} % except the center
\renewcommand{\headrulewidth}{0pt}
\renewcommand{\footrulewidth}{0pt}}

\theoremstyle{plain}
\newtheorem{theorem}{Theorem}
\newtheorem{assumption}{Assumption}
\newtheorem{corollary}{Corollary}
\newtheorem{lemma}{Lemma}
\newtheorem{conjecture}{Conjecture}
\newtheorem{proposition}{Proposition}
\newtheorem{observation}{Observation}
\newtheorem{claim}{Claim}
\newtheorem{property}{Property}
\newtheorem{op}{Open Problem}
\newtheorem{problem}{Problem}
\newtheorem{question}{Question}

\theoremstyle{definition}
\newtheorem{definition}{Definition}
\newtheorem{example}{Example}
\newtheorem{sketch}{Sketch}
\newtheorem{idea}{Idea}

\theoremstyle{remark}
\newtheorem{remark}{Remark}

\newtheoremstyle{restate}{}{}{\itshape}{}{\bfseries}{~(restated).}{.5em}{\thmnote{#3}}
\theoremstyle{restate}
\newtheorem*{restate}{}



\crefname{theorem}{Theorem}{Theorems}
\crefname{assumption}{Assumption}{Assumptions}
\crefname{corollary}{Corollary}{Corollaries}
\crefname{lemma}{Lemma}{Lemmas}
\crefname{conjecture}{Conjecture}{Conjectures}
\crefname{proposition}{Proposition}{Propositions}
\crefname{observation}{Observation}{Observations}
\crefname{claim}{Claim}{Claims}
\crefname{property}{Property}{Properties}
\crefname{op}{Open Problem}{Open Problems}
\crefname{problem}{Problem}{Problems}
\crefname{question}{Question}{Questions}

% \crefname{fact}{Fact}{Facts}

\crefname{definition}{Definition}{Definitions}
\crefname{example}{Example}{Examples}
\crefname{sketch}{Sketch}{Sketches}
\crefname{idea}{Idea}{Ideas}

% \crefname{condition}{Condition}{Conditions}

\crefname{remark}{Remark}{Remarks}



\crefname{equation}{Equation}{Equations}
\crefname{figure}{Figure}{Figures}
\crefname{table}{Table}{Tables}



%%%%%%%%%%%%%%%%%%%%%%%%%%
% GENERAL-PURPOSE MACROS %
%%%%%%%%%%%%%%%%%%%%%%%%%%
\newcommand{\uth}{\bigskip \bigskip {\huge {\red{UP TO HERE}} \bigskip \bigskip}}
\newcommand{\ignore}[1]{}
\newcommand{\eps}{\varepsilon}
\newcommand{\simple}{\mathrm{simple}}
\newcommand{\E}{\operatorname{{\bf E}}}
\newcommand{\Ex}{\mathop{{\bf E}\/}}
\renewcommand{\Pr}{\operatorname{{\bf Pr}}}
\newcommand{\Prx}{\mathop{{\bf Pr}\/}}
\newcommand{\Var}{\operatorname{{\bf Var}}}
\newcommand{\Varx}{\mathop{{\bf Var}\/}}
\newcommand{\tO}{\tilde{O}}
\newcommand{\sgn}{\mathrm{sgn}}
\newcommand{\X}{\mathcal{X}}
\newcommand{\Y}{\mathcal{Y}}
\newcommand{\rZ}{\mathcal{Z}}

\DeclareMathOperator\erf{erf}

\newcommand{\polylog}{\mathrm{polylog}}
\newcommand{\poly}{\mathrm{poly}}
\newcommand{\bcalZ}{\bm{\mathcal{Z}}}
\newcommand{\bx}{\bm{x}}
\newcommand{\by}{\bm{y}}
\newcommand{\bxi}{\bm{\xi}}

%%%%%%%%%%%%%%%%%%
% NUMBER SYSTEMS %
%%%%%%%%%%%%%%%%%%
\newcommand{\R}{\mathbb R}
\newcommand{\RR}{\R_{\geq 0}}
\newcommand{\C}{\mathbb C}
\newcommand{\N}{\mathbb N}
\newcommand{\NN}{\N_{\geq 1}}
\newcommand{\Z}{\mathbb Z}

\renewcommand{\i}{\mathbf{i}}   % for complex numbers
\renewcommand{\d}{\mathrm{d}}   % for integrals
\newcommand{\lhs}{\mathrm{LHS}} % for inequalities
\newcommand{\rhs}{\mathrm{RHS}} % for inequalities
\newcommand{\supp}{\mathrm{supp}}
\renewcommand{\hat}[1]{\widehat{#1}}
\renewcommand{\bar}[1]{\overline{#1}}
\newcommand{\sig}{\mathrm{sig}}

\newcommand{\comment}[1]{}

% Define colors
\def\colorful{1}
\ifnum\colorful=1
\newcommand{\violet}[1]{{\color{violet}{#1}}}
\newcommand{\orange}[1]{{\color{orange}{#1}}}
\newcommand{\blue}[1]{{{\color{blue}#1}}}
\newcommand{\red}[1]{{\color{red} {#1}}}
\newcommand{\green}[1]{{\color{green} {#1}}}
\newcommand{\pink}[1]{{\color{pink}{#1}}}
\newcommand{\gray}[1]{{\color{gray}{#1}}}
\fi
\ifnum\colorful=0
\newcommand{\violet}[1]{{{#1}}}
\newcommand{\orange}[1]{{{#1}}}
\newcommand{\blue}[1]{{{#1}}}
\newcommand{\red}[1]{{{#1}}}
\newcommand{\green}[1]{{{#1}}}
\newcommand{\gray}[1]{{{#1}}}
\fi

\title{Homework 3}
% \author{Tim Randolph}
\date{COMS W3261, Summer B 2021}

\begin{document}

\maketitle

This homework is due \textbf{Monday, 7/19/2021, at 11:59PM EST}. Submit to GradeScope (course code: X3JEX4).

Grading policy reminder: \LaTeX~is preferred, but neatly typed or handwritten solutions are acceptable. Your TAs may dock points for indecipherable writing. Proofs should be complete; that is, include enough information that a reader can clearly tell that the argument is rigorous.

The tool \url{http://madebyevan.com/fsm/} may be useful for drawing finite state machines.

If a question is ambiguous, please state your assumptions. This way, we can give you credit for correct work. (Even better, post on Ed so that we can resolve the ambiguity.)



\clearpage
\section{Problem 1 (12 points)}

\begin{enumerate}
    \item (6 points.) Prove that the language
    \[
        A = \{w \; | \; \text{ For all } y \in \{0, 1\}^*, w \neq yy\}
    \]
    over the alphabet $\Sigma = \{0,1\}$ is nonregular. You may use the pumping lemma and/or closure properties.
    
    \item (6 points.) Prove that the language
    \[
        B = \{1^n0^m1^n \; | \; n \geq 0, m \geq 1\}
    \]
    over the alphabet $\Sigma = \{0,1\}$ is nonregular. You may use the pumping lemma and/or closure properties.
\end{enumerate}

\clearpage
\section{Problem 2 (8 points)}
\begin{enumerate}
    \item (8 points.) Is the language
    \[
        C = \{a^ib^jc^k \; | \; i, j, k \geq 0; i \leq j\} \cup
        \{a^ib^jc^k \; | \; i, j, k \geq 0; j \leq k\} \cup
        \{a^ib^jc^k \; | \; i, j, k \geq 0; k \leq i\}
    \]
    over the alphabet $\Sigma = \{a, b, c\}$ a regular language? Prove your answer.
\end{enumerate}

\clearpage
\section{Problem 3 (10 points)}
    \begin{enumerate}
        \item (4 points). Convert the DFA below into a GNFA state diagram using the procedure outlined in class. (This procedure is also outlined in the textbook on page 71.)
        
        \begin{center}
        \begin{tikzpicture}[scale=0.2]
        \tikzstyle{every node}+=[inner sep=0pt]
        \draw [black] (29.7,-27.5) circle (3);
        \draw (29.7,-27.5) node {$q_0$};
        \draw [black] (41.2,-18.8) circle (3);
        \draw (41.2,-18.8) node {$q_1$};
        \draw [black] (41.2,-18.8) circle (2.4);
        \draw [black] (41.2,-36) circle (3);
        \draw (41.2,-36) node {$q_2$};
        \draw [black] (41.2,-36) circle (2.4);
        \draw [black] (30.388,-24.593) arc (157.75081:96.46579:9.627);
        \fill [black] (38.21,-18.67) -- (37.36,-18.26) -- (37.48,-19.26);
        \draw (32.49,-20.06) node [above] {$0$};
        \draw [black] (38.224,-36.257) arc (-94.37356:-158.56491:9.24);
        \fill [black] (38.22,-36.26) -- (37.46,-35.7) -- (37.39,-36.69);
        \draw (32.43,-34.97) node [below] {$1$};
        \draw [black] (23.9,-27.5) -- (26.7,-27.5);
        \fill [black] (26.7,-27.5) -- (25.9,-27) -- (25.9,-28);
        \draw [black] (39.969,-21.528) arc (-31.2107:-74.57271:12.405);
        \fill [black] (32.66,-27.06) -- (33.56,-27.33) -- (33.3,-26.36);
        \draw (37.85,-25.49) node [below] {$0$};
        \draw [black] (44.073,-17.978) arc (133.69515:-154.30485:2.25);
        \draw (48.57,-20.27) node [right] {$1$};
        \fill [black] (43.6,-20.58) -- (43.79,-21.5) -- (44.51,-20.81);
        \draw [black] (43.88,-34.677) arc (144:-144:2.25);
        \draw (48.45,-36) node [right] {$0,1$};
        \fill [black] (43.88,-37.32) -- (44.23,-38.2) -- (44.82,-37.39);
        \end{tikzpicture}
        \end{center}
        
        \item (6 points). Use the procedure $CONVERT(G)$ outlined in class (and on page 73 of the textbook) to compute the values of the transitions $\delta'(q_{start}, q_2)$, $\delta'(q_{start}, q_{accept})$, and $\delta'(q_2, q_{accept})$ after removing state $q_1$ from the GNFA below. Hint: Recall that $\emptyset^*$ evaluates to the language $\{\varepsilon\}$.
        
        \begin{center}
        \begin{tikzpicture}[scale=0.2]
        \tikzstyle{every node}+=[inner sep=0pt]
        \draw [black] (19,-26.9) circle (3);
        \draw (19,-26.9) node {$q_{start}$};
        \draw [black] (38.7,-15.6) circle (3);
        \draw (38.7,-15.6) node {$q_1$};
        \draw [black] (38.7,-15.6) circle (2.4);
        \draw [black] (38.7,-31.3) circle (3);
        \draw (38.7,-31.3) node {$q_2$};
        \draw [black] (58.2,-26.9) circle (3);
        \draw (58.2,-26.9) node {$q_{accept}$};
        \draw [black] (58.2,-26.9) circle (2.4);
        \draw [black] (12.7,-26.9) -- (16,-26.9);
        \fill [black] (16,-26.9) -- (15.2,-26.4) -- (15.2,-27.4);
        \draw [black] (20.209,-24.159) arc (151.19528:88.48224:17.185);
        \fill [black] (35.72,-15.26) -- (34.94,-14.74) -- (34.91,-15.74);
        \draw (25.26,-17.03) node [above] {$11$};
        \draw [black] (57.329,-29.768) arc (-21.16737:-158.83263:20.084);
        \fill [black] (57.33,-29.77) -- (56.57,-30.33) -- (57.51,-30.69);
        \draw (38.6,-43.1) node [below] {$\emptyset$};
        \draw [black] (37.909,-28.408) arc (-168.24103:-191.75897:24.329);
        \fill [black] (37.91,-28.41) -- (38.24,-27.52) -- (37.26,-27.73);
        \draw (36.9,-23.45) node [left] {$0^*$};
        \draw [black] (40.001,-18.298) arc (20.03358:-20.03358:15.04);
        \fill [black] (40,-18.3) -- (39.81,-19.22) -- (40.74,-18.88);
        \draw (41.41,-23.45) node [right] {$0^+$};
        \draw [black] (41.678,-15.272) arc (91.24006:28.57645:17.032);
        \fill [black] (57,-24.15) -- (57.06,-23.21) -- (56.18,-23.69);
        \draw (52.25,-17.06) node [above] {$\Sigma\Sigma$};
        \draw [black] (41.63,-30.64) -- (55.27,-27.56);
        \fill [black] (55.27,-27.56) -- (54.38,-27.25) -- (54.6,-28.22);
        \draw (49.05,-29.67) node [below] {$\varepsilon$};
        \draw [black] (21.93,-27.55) -- (35.77,-30.65);
        \fill [black] (35.77,-30.65) -- (35.1,-29.98) -- (34.88,-30.96);
        \draw (29.81,-28.48) node [above] {$10$};
        \draw [black] (40.023,-33.98) arc (54:-234:2.25);
        \draw (38.7,-38.55) node [below] {$1\cup0$};
        \fill [black] (37.38,-33.98) -- (36.5,-34.33) -- (37.31,-34.92);
        \draw [black] (37.377,-12.92) arc (234:-54:2.25);
        \draw (38.7,-8.35) node [above] {$\emptyset$};
        \fill [black] (40.02,-12.92) -- (40.9,-12.57) -- (40.09,-11.98);
        \end{tikzpicture}
        \end{center}
    \end{enumerate}
    
    
    
\clearpage
\section{Problem 4 (12 points)}
    \begin{enumerate}
        \item (3 points). What is the language of the grammar $G_1$ below? Here $S$, $A$, and $B$ are the variables and $0$ and $1$ are the terminals. Explain your reasoning.
        \begin{align*}
            S &\rightarrow 1A1 \\
            A &\rightarrow S \; | \; B \\
            B &\rightarrow 0B \; | \; \varepsilon
        \end{align*}
        
        \item (3 points). What is the language of the grammar $G_2$ below? Here $A$ and $B$ are the variables and $x$, $y$, and $z$ are the terminals. Explain your reasoning.
        \begin{align*}
            A &\rightarrow xAx \; | \; yAy \; | \; zAz \; | \; B \\
            B &\rightarrow x \; | \; y \; | \; z \; | \; \varepsilon
        \end{align*}
        
        \item (3 points). Design a grammar for the language
        \[
            D = \{a^i b^j c a^j b^i \; | \; i, j \geq 1\}
        \]
        and explain why your grammar produces $D$.
        
        \item (3 points). Design a grammar for the language
        \[
            L = I(saw \cup met \cup loved)(the \cup a)(very)^*(large \cup tiny \cup red)(frog \cup dog)
        \]
        and explain why your grammar produces $L$. (You can treat each word as a single terminal symbol.)
    \end{enumerate}

\end{document}
